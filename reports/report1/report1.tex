\documentclass[a4paper, 12pt]{article}

\title{Report 1}
\author{Ali Abolhassanzadeh Mahani}

\begin{document}
	\maketitle
	
	\section{Progress}
	\begin{itemize}
		\item Initialzing the system: the size, the population of the neuron, (Are they constant or do they grow?)
		\item The initial position of the neurons is chosen randomly within a square canvas with side \texttt{size}.
		\item 
	\end{itemize}
	\section{Problems}
	\begin{itemize}
		\item I have questions regarding the time scales introduced in the article. What is $\tau$? How do I work with it?
		\item How should I simulate the firing rate of the neurons? And what do they mean in the model?
		\item I working on the equation for the mutual area of two circles a distaance $d$ apart.
	\end{itemize}

	\section{Ideas}
	\begin{itemize}
		\item we can implement the system in 3D as well. The factor of dimension can influence the system. specially if it's a critical one.
		\item Boundary conditions. In the article, the canvas is blocked at all sides. The critical system can 
		be influenced by the system size. If we take the boundary conditions to be periodic, the system may act
		a bit different. For example, The over shoot in the neuron growth in the article, is mostly seen in the 
		surrounding neurons. This might be changed if we devise periodic boundary conditions.
	\end{itemize}
\end{document}